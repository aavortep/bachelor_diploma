\specsection{Заключение}

В результате работы были рассмотрены понятия предметной области, такие как темп и ритм музыки, и проанализирована проблема автоматического определения темпа и ритма.

Были определены критерии сравнения методов.

Были рассмотрены основные методы автоматического определения темпа и ритма музыки: дискретное вейвлет-преобразование, скрытые марковские модели, байесовское иерархическое моделирование и сверточные нейронные сети. После чего было произведено сравнение изученных методов по выделенным ранее критериям.

Таким образом, цель работы -- изучить основные существующие методы определения ритмического рисунка и темпа цифровой музыкальной записи -- была достигнута.