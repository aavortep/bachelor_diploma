\specsection{ЗАКЛЮЧЕНИЕ}

В результате работы были рассмотрены понятия предметной области, такие как темп и ритм музыки, и проанализирована проблема автоматического определения темпа и ритма.

Были определены критерии сравнения методов.

Были рассмотрены основные методы автоматического определения темпа и ритма музыки: дискретное вейвлет-преобразование, скрытые марковские модели, байесовское иерархическое моделирование и сверточные нейронные сети. После чего было произведено сравнение изученных методов по выделенным ранее критериям.

По результатам сравнения можно сделать вывод, что ни один из рассмотренных методов без каких-либо модификаций не позволяет определять переменный темп и ритм. При этом сверточные нейросети позволяют добиться достаточно высокой точности результатов в сравнении с другими методами, не имея особых ограничений на формат входного аудиофайла, но для этого требуется выборка достаточно больших размеров и значительное время на обучение.

Таким образом, цель работы -- изучить основные существующие методы определения ритмического рисунка и темпа цифровой музыкальной записи -- была достигнута.