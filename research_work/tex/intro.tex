\specsection{ВВЕДЕНИЕ}

Автоматическая транскрипция музыки (АТМ) — это процесс преобразования акустического музыкального сигнала в ту или иную форму нотной записи~\cite{future_dir}. Данную задачу можно разделить на несколько подзадач, к которым в том числе относятся задачи выделения информации о ритме и темпе музыки. Несмотря на то, что задачу АТМ для монофонических сигналов можно считать решенной~\cite{future_dir}, проблема создания автоматизированной системы, способной транскрибировать полифоническую (многоголосую) музыку без ограничений по степени полифонии или типу инструмента, остается открытой.

Цель данной работы – изучить основные существующие методы определения ритмического рисунка и темпа цифровой музыкальной записи.

Чтобы достигнуть поставленной цели, требуется решить следующие задачи:
\begin{itemize}
	\item[---] провести анализ предметной области и сформулировать проблему;
	\item[---] сформулировать критерии сравнения методов выделения информации о ритме и темпе музыки;
	\item[---] классифицировать основные существующие методы.
\end{itemize}