\section{Аналитическая часть}

\subsection{Темп, ритм и метр}

\textbf{Темп} -- мера времени в музыке, упрощенно -- «скорость исполнения музыки» \cite{grouv}.

Существует несколько способов измерения темпа. В классической музыке чаще всего используется словесное описание (как правило, на итальянском). Этот метод является неточным и дает лишь примерное представление о <<скорости>> исполнения музыкального произведения. Примеры такого описания: адажио, ленто (медленные темпы); анданте, модерато (средние темпы); аллегро, виво (быстрые темпы).

Второй, более точный способ измерения темпа -- это число ударов в минуту (beats per minute, сокращенно bpm). Данный метод напрямую связан с частотой колебания маятника в метрономе (устройстве, предназначенном для точного ориентира темпа при исполнении музыки). Стандартным темпом считается 120 bpm, т. е. 2 Гц.

В данной работе будет использоваться второй способ измерения темпа (в bpm).

\textbf{Ритм} -- организация музыки во времени \cite{chehovich}. Ритмическую структуру музыки образует последовательность длительностей -- звуков и пауз.

Ритм в музыке принадлежит к числу терминов, дискутируемых в науке последних двух столетий. Единого мнения по вопросу его определения нет. Чаще всего ритм определяется как регулярная, периодическая последовательность акцентов. Такое понимание ритма фактически идентично метру.

\textbf{Метр} в музыке -- это чередование сильных и слабых долей в определенном темпе \cite{grouv}. Обычно метр фиксируется с помощью тактового размера и тактовой черты. Размер задаёт относительную длительность каждой доли. Например, размер <<3/4>> говорит о том, что в такте 3 доли, каждая из которых представлена четвертной нотой. Можно сказать, что размер -- числовое представление метра с указанием длительности каждой доли. Такт в свою очередь -- единица метра, начинающаяся с наиболее сильной доли и заканчивающаяся перед следующей равной ей по силе.

В данной работе не будут учитываться тонкости различия ритма и метра. Соответственно, для измерения ритма будет использоваться числовое представление метра в виде тактового размера.

\subsection{Проблема определения ритма и темпа}

Из всего вышесказанного следует, что для определения ритма необходимо выявить сильные и слабые доли, их количество и закономерность их чередования. В свою очередь, для определения темпа необходимо выявить частоту чередования этих долей.

Помимо этого, в музыкальных записях с живыми инструментами есть некоторые особенности, которые затрудняют определение ритма и темпа. Одна из них -- это нечеткое попадание инструмента в ритмическую сетку. Такие небольшие отклонения на живых записях присутствуют всегда. Они не заметны для уха человека, но могут осложнять автоматическое распознавание.

Также в некоторых случаях темп может изменяться в течение музыкального произведения.

\subsection{Дискретное вейвлет-преобразование}



\subsection{Скрытые модели Маркова}



\subsection{Байесовское иерархическое моделирование}



\subsection{Использование сверточных нейросетей}



\subsection{Сравнение методов}



\subsection*{Выводы}

В этом разделе была проанализирована предметная область и обозначена проблема. А также была проведена классификация и сравнение основных существующих методов решения поставленной задачи.


