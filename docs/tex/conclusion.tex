\specsection{ЗАКЛЮЧЕНИЕ}

В ходе выполнения выпускной квалификационной работы были рассмотрены методы автоматического определения ритмического рисунка и темпа цифровой музыкальной записи (ДВП, скрытые модели Маркова, байесовское иерархическое моделирование, сверточные нейросети), проведен обзор существующих решений, приведены результаты сравнительного анализа.

Был разработан метод определения переменного ритмического рисунка и переменного темпа цифровой музыкальной записи на основе байесовского иерархического моделирования.

Было разработано программное обеспечение, реализующее описанный метод, выполнено его тестирование, описан пользовательский интерфейс.

После этого было проведено исследование применимости разработанного программного обеспечения на музыке с разным составом инструментов и разных жанров. А также выполнено сравнение результатов работы реализованного метода с результатами, полученными с помощью библиотечной функции.

Таким образом, поставленная цель -- реализовать метод автоматического определения темпа и ритма музыки на основе байесовского иерархического моделирования -- была достигнута.