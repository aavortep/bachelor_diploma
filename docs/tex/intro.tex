\section*{ВВЕДЕНИЕ}\addcontentsline{toc}{section}{ВВЕДЕНИЕ}

Автоматическая транскрипция музыки (АТМ) — это процесс преобразования акустического музыкального сигнала в ту или иную форму нотной записи~\cite{future_dir}. АТМ чаще всего используется музыкантами для получения нот или табулатур по аудиофайлу с целью дальнейшего изучения этой композиции. Данную задачу можно разделить на несколько подзадач, к которым в том числе относятся задачи выделения информации о ритме и темпе музыки. Несмотря на то, что задачу АТМ для монофонических сигналов можно считать решенной~\cite{future_dir}, проблема создания автоматизированной системы, способной транскрибировать полифоническую (многоголосую) музыку без ограничений по степени полифонии или типу инструмента, остается открытой.

Цель данной работы – реализовать метод автоматического определения темпа и ритма музыки на основе байесовского иерархического моделирования.

Чтобы достигнуть поставленной цели, требуется решить следующие задачи:
\begin{itemize}
	\item[---] провести анализ предметной области и сформулировать проблему;
	\item[---] проанализировать и сравнить основные методы определения темпа и ритма;
	\item[---] разработать метод решения поставленной задачи;
	\item[---] спроектировать архитектуру разрабатываемого программного обеспечения;
	\item[---] реализовать разработанный метод;
	\item[---] протестировать и сравнить результаты работы реализованного метода с результатами, полученными с помощью известных аналогов.
\end{itemize}